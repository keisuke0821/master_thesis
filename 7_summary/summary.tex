\chapter{まとめ}
\label{chapter6}

CMBの偏光観測は宇宙の進化を説明するための鍵となっており、多くの観測実験が進められている。特に$\ell\sim 10$での大角度スケールのCMB偏光パターンは宇宙の再電離期の情報が刻まれ、その観測によってニュートリノ質量和の精密測定に寄与できる。GroundBIRD実験は大角度スケールのCMB偏光の観測に特化した地上CMB望遠鏡である。望遠鏡を最大で1分間で20回転させる独自のスキャン戦略をとることで、地上実験にとって障壁となる大気放射の揺らぎを抑制した観測を実現する。時間応答性の良い超伝導検出器MKIDを焦点面検出器として採用し、高速スキャンに伴う高いサンプリングレートを可能にしている。2023年5月から、GroundBIRDは本格的な観測を開始した。

目標とする光学的厚み$\tau$の測定には3年間の観測を実施し、統計量を貯める必要があるため、望遠鏡の観測システムは安定して長期運用ができること、そして質の良いデータを取得し続けることが必要になる。しかし、本研究の開始前の観測システムにおいて2つの未解決課題があった。1つは望遠鏡仰角データ取得システムが硬直的であること、もう1つは天球上での検出器配置がスキャン軸から傾いていることである。本論文ではこれらの課題に対する改善と最適化を行なった。

仰角データの取得にFPGAボードを使用しており、FPGAチップ内でデータ処理を行っている。既存システムではその運用をリモート主体で行えず、現地でのメンテナンスを要する点で長期運用の障壁となっていた。本論文ではボード内のFPGAチップにPYNQと呼ばれるOSシステムを搭載し、アクセス性の向上とOS上からソフトウェアを動かすことでデータ取得システムの操作性向上を図った。また、信号処理の確認と安定動作の確認を行った後、望遠鏡システムへのインストールを完了させた。

検出器MKIDはCMBの偏光信号と大気放射由来のノイズを検出するが、スキャン軸上の異なる検出器間で信号の差分をとることで、共通した大気ノイズを差し引くことができる。しかし、配置が傾いていると検出器間で観測する大気が揺らぎ、差分をとってもノイズが残ってしまうため、データの質が落ちてしまう。そこで、天球上での検出器配置が望遠鏡のスキャン軸に対して約$\SI{6}{^{\circ}}$と有意に傾いていることを月の観測データから見積もった。この結果をもとに、望遠鏡を視線方向軸の周りに回転させることで天球上での検出器配置の改善を施し、さらに月と木星の観測データを用いて改善を確認した。加えて、スキャン軸上の検出器間で信号の差分をとり、検出器間での相関の強さを示す指標に焼き直し、回転の前後で比較することで観測する大気の揺らぎを抑制する結果を得た。

以上2点の改善と最適化を通してGroundBIRDが持つ観測、運用性能を向上させることに成功した。
