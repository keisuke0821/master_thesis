\begin{abstract}
\hspace{-4mm}

%gaiyou(CMBで迫りたい物理(ニュートリノ質量和、tau)、GB、自分のやったことがどう重要なのかを簡潔に記す)
宇宙マイクロ波背景放射(CMB)の温度異方性の観測によって宇宙の進化を記述する標準理論が構築されてきた。現在ではCMBの偏光観測が大きなテーマとなっており、偏光を通してインフレーション理論やニュートリノ質量和といった未だ解明されていない課題に迫ることができると期待されている。

GroundBIRDはスペイン領テネリフェ島の標高2,400 mに位置する大角度スケールの観測に特化した地上CMB望遠鏡である。望遠鏡を1分間で20回転させる高速スキャンを実現することで大気揺らぎの影響を抑制した観測を行う。時間応答性の良い超伝導検出器MKIDを採用し、2023年5月に全焦点面検出器のインストールを完了させ、本格的な物理観測を開始した。大角度スケールでの偏光Eモードを観測することで、宇宙再電離期を特徴付けるパラメータである光学的厚み$\tau$を誤差$\sigma_{\tau}\sim 0.01$で測定することを目指している。

観測を続けてデータを蓄積する段階にある現在、安定して長期運用をすること、そして質の良いデータを取得することが要求される。しかし、既存の観測システムにおいて、望遠鏡仰角データ取得システムが硬直的である問題と天球上での検出器配置がスキャン軸から傾いている問題があり、この要求を十分に満たせていない。本論文ではこれらの問題に対する改善と最適化を行なった。

仰角のデータ取得にFPGAボードを使用しているが、そのシステムの運用をリモート主体で行えず、現地でのメンテナンスを要するという点で長期運用に対する障壁となっていた。本論文ではボード内のFPGAチップにPYNQと呼ばれるOSシステムを搭載し、アクセス性とOS上からソフトウェアを動かすことでデータ取得の操作性向上を図った。また、信号処理の確認と安定動作の確認を行った後、実際の望遠鏡システムへのインストールを完了させた。

天球上での検出器配置が望遠鏡のスキャン軸に対して約6°傾いていることを月の観測データから見積もった。検出器MKIDはCMBの偏光信号と大気放射由来のノイズを検出するが、スキャン軸上の異なる検出器間で信号の差分をとることで、共通した大気ノイズを差し引くことができる。しかし、配置が傾いていると検出器間で観測する大気が揺らぎ、差分をとってもノイズが残ってしまい、データの質が落ちてしまう。そのため、望遠鏡を視線方向軸の周りに回転させることで天球上での検出器配置を改善した。回転による配置の変化を月と木星の観測データを用いて確認した。加えて、スキャン軸上の検出器間で信号の差分をとり、検出器間での相関の強さを示す指標に焼き直し、回転の前後で比較することで観測する大気の揺らぎを抑制する結果を得た。

以上2点の改善と最適化を通してGroundBIRDが持つ運用、観測性能の向上を成功させた。



%本論文の構成を述べる。第\ref{chapter1}章でCMBに関わる理論的な背景、第\ref{chapter2}章でGroundBIRD実験の概要を説明する。以降は第\ref{chapter3}章と第\ref{chapter4}章の2部構成になっており、第\ref{chapter3}章でGroundBIRDの角度データ取得システムの改善について、第\ref{chapter4}章で焦点面検出器のアライメント較正とその結果について述べる。第\ref{chapter5}章で今後の展望を述べ、第\ref{chapter6}章でまとめを述べる。


\end{abstract}
