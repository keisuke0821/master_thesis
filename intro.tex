\documentclass[11pt]{jsreport}

\usepackage[utf8]{inputenc}
\usepackage{amsmath,amssymb}
\usepackage{latexsym}
\usepackage{bm}
\usepackage{braket}
\usepackage{otf}
\usepackage{tikz}
\usepackage{url}
\usepackage{textcomp, gensymb}
\usepackage{siunitx}
\usepackage{amsmath,amsfonts,amssymb}
\usepackage{subcaption}
\usepackage{here}
\usepackage{mathtools}
\usepackage{tikz}
\usepackage{multirow}
\usepackage[dvipdfmx]{hyperref}
\usepackage{pxjahyper}
\hypersetup{%
 setpagesize=true,%
 bookmarks=true,%
 bookmarksdepth=tocdepth,%
 bookmarksnumbered=true,%
 colorlinks=false,%
}

\begin{document}
\begin{titlepage}
  \begin{center}
    {\Huge 修士学位論文}

    \vspace{50truept}

    {\Huge タイトル}

    \vspace{80truept}

    {\LARGE 片岡 敬涼}

    \vspace{50truept}

    {\LARGE 京都大学 理学研究科 物理学・宇宙物理学専攻 \\ 物理学第二教室 高エネルギー物理学研究室}

    \vspace{100truept}

    {\LARGE 2025年2月1日}

  \end{center}
\end{titlepage}

\clearpage

\begin{center}
  {\Large 概要}
\end{center}

概要をこのページに書く。文字のサイズがこれでいいのかはよくわからんけど、一旦こんな感じでやっていけばいいか。

\clearpage

\tableofcontents

\chapter{序論}
\section{aa}
あいうえお
\subsection{aaa}
あいうえおかきくけこ
\subsection{aaa}
あいうえお

\chapter{CMB望遠鏡のデータ取得}
\section{bb}
あいうえお
\subsection{cc}
かきくけこ
\subsection{aa}
あいうえお

\chapter{hoge}
\section{aa}
あいうえお
\subsection{aa}
あいうえお

\chapter*{謝辞}
ありがとう

\appendix
\chapter{付録}
おまけ

\begin{thebibliography}{99}
  \bibitem{hoge} 教科書
  \bibitem{huga} 教科書
\end{thebibliography}

\end{document}
