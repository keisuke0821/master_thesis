\chapter{謝辞}

京都大学高エネルギー物理学研究室で過ごした2年間の研究生活は大変有意義なもので、多くのことを学び、成長することができたと実感しています。本論文の執筆に至るまでご指導・ご支援いただいた全ての方々に感謝を申し上げます。

田島治教授にはCMBの研究を始める入り口を作っていただき、充実した研究をするきっかけを与えてくださいました。また、ミーティングなどで的確なアドバイスをいただき、研究に対する理解を深めることができました。そして、修士論文を執筆するまで計2回のテネリフェへの出張をすることができたのは田島さんの多大なる協力のお陰です。ご多忙の中、修士論文の添削に多くのお時間を割いていただきありがとうございました。

鈴木惇也助教とは1年間同室で過ごし、多くのご指導とアドバイスをいただきました。日頃から気にかけていただき、基礎的なことから実験の詳細のことまで丁寧に教えていただきました。修士一年目の研究が思うように進まなかった時期には、夜遅くまで付きっきりでご指導していただき、心から感謝しています。

GroundBIRD実験の皆様にも大変お世話になりました。末野慶徳氏には実験に関する全ての面でたくさんのことを教えていただきました。テネリフェでの作業時にも通話を繋いで直接アドバイスをいただきました。末野さんがグループを引っ張る姿を追いかけながら私自身も成長することができました。武市宗一郎氏には解析面で多くの助言をいただきました。解析手法を共有していただいたことで自分の理解をより深めることができました。東北大学の田中智永氏には京都で解析について基本的なことから丁寧に教えていただき、感謝しています。東北大学の辻井未来氏とIACのAlessandro Fasano氏にテネリフェでの作業や生活面でも本当にお世話になりました。初海外で右も左も分からなかった私をサポートしていただきました。ここには書ききれませんが、全てのGroundBIRDメンバーの方々へ、2年間のびのびと研究できたのは皆様のご協力あってのことであり、感謝の気持ちでいっぱいです。
%同期の笠井優太郎くんは優しくていい人でした。
