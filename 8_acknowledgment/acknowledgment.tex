\chapter{謝辞}

本修士論文の執筆にあたり、多くの方々にご支援いただきました。
ご支援いただいた全ての方々にこの場を借りてお礼申し上げます。
この~2~年間は非常に有意義なもので、大いに成長できたと思います。


指導教員である田島治准教授には、CMB~という魅力的なテーマと、研究を行う上でこれ以上ない環境を作っていただきました。
研究も基本的には自由にやらせていただいて、とても楽しく、やりがいがありました。
また、数多くの的確な助言などで研究を正しい道に導いていただきました。
修士論文の作成にも多大なる時間と労力を割いていただきました。
文章を書くのが下手な私も無事に修士論文を書き終えることができたのは、田島さんの尽力のおかげです。
本当にありがとうございます。

GroundBIRD~実験のメンバーの皆様にも感謝申し上げます。
鈴木惇也助教にはデバイス扱い方から本修士論文の内容まで、私の研究のほぼ全てにおいてサポートしていただきました。
研究が詰まった時は、よく相談にも乗っていただき、ミーティングでも毎回的確な助言をいただきました。
本多俊介氏には読み出し回路や~GroundBIRD~実験について数多く教えをいただきました。
夜遅くまで私の研究を手伝っていただくこともありました。
また、よくコーヒーにも誘っていただき、良いリフレッシュをすることができました。
東北大学の沓間弘樹氏には~MKID~について非常に多くのことを教えていただきました。
修士論文の内容についても相談に乗っていただき、多くの助言をいただきました。
%良いものになったのも沓間さんの力が大きいです。
どんな質問でも嫌がらず、丁寧に説明してくれました。
私の~MKID~の基礎は沓間さんによって作られたと言っても過言ではありません。
池満拓司氏には~MKID~の測定手法の基本を教わりました。
%池満さんの着実に研究を進める様子は非常に魅力的でした。
また、テネリフェでは海に連れて行っていただくなど、テネリフェでの私生活でもお世話になりました。
小栗秀悟助教、長崎岳人氏にはテネリフェで~GroundBIRD~望遠鏡について様々なことを教わりました。
理研の美馬覚氏には冷凍機と~MKID~について助言をいただきました。

京都~CMB~グループの皆様にもお世話になりました。
安達俊介氏には冷凍機の立ち上げからその運用に関して多くの助言を多くいただきました。
大塚稔也君には研究を行う上で毎日、非常に刺激を受けました。
先々のことを見据え、着実に研究を進める大塚君が同期にいたことは私の研究姿勢にも良い影響を及ぼしてくれました。
阿部倫史氏と小高駿平君、中田嘉信君にはミーティングやゼミでお世話になりました。

デルフト工科大学の遠藤光氏、SRON~の唐津謙一氏、埼玉大学の成瀬雅人氏には
MKID~の評価系を構築する上で多くの助言をいただきました。心から感謝申し上げます。

ここには書ききれませんが、研究生活の様々な場面でお世話になった先生方、先輩、後輩、秘書の皆様にも感謝申し上げます。
%のびのびと研究のできる良い研究室に恵まれたと実感しております。
また、同期の小林蓮君、菅島文悟君、谷真央君、辻川吉明君は一緒に博士課程に進学することもあり、多くの刺激をうけました。

最後に、いつも応援し、支えてくれた家族と真衣に感謝します。ありがとう。