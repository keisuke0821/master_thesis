\chapter{まとめ}
\label{chapter6}

CMBの偏光観測は宇宙の進化を説明するための鍵となっており、多くの観測実験が進められている。特に$\ell\sim 10$での大角度スケールのCMB偏光パターンは宇宙の再電離期の情報が刻まれ、その観測によってニュートリノ質量和の精密測定に寄与する重要なプローブである。GroundBIRD実験は大角度スケールのCMB偏光の観測に特化した地上CMB望遠鏡である。望遠鏡を最大で1分間で20回転させる独自のスキャン戦略をとることで、地上実験にとって障壁となる大気放射の揺らぎを抑制した観測を実現する。時間応答性の良い超伝導検出器MKIDを採用し、高速スキャンに伴う高いサンプリングレートを可能にしている。2023年5月に全焦点面検出器のインストールが完了し、GroundBIRDは本格的な物理観測の段階に突入した。

目標とする光学的厚み$\tau$の測定には3年間の観測を実施して統計量を貯める必要があるため、望遠鏡の観測システムは安定して長期運用ができること、そして質の良いデータを取得し続けることが必要になる。しかし、望遠鏡仰角データ取得システムが硬直的である問題と天球上での検出器配置がスキャン軸から傾いている問題があり、この要求を十分に満たせていない状況にあった。本論文ではこれらの問題に対する改善と最適化を行なった。

仰角のデータ取得にFPGAボードを使用しており、FPGAチップ内でデータ処理を行っている。既存システムではその運用をリモート主体で行えず、現地でのメンテナンスを要する点で長期運用の障壁となっていた。本論文ではボード内のFPGAチップにPYNQと呼ばれるOSシステムを搭載し、アクセス性の向上とOS上からソフトウェアを動かすことでデータ取得システムの操作性向上を図った。また、信号処理の確認と安定動作の確認を行った後、望遠鏡システムへのインストールを完了させた
