\begin{abstract}
\hspace{-4mm}

%gaiyou(CMBで迫りたい物理(ニュートリノ質量和、tau)、GB、自分のやったことがどう重要なのかを簡潔に記す)
宇宙マイクロ波背景放射(CMB)の温度異方性の観測によって、宇宙の進化を記述する標準理論が構築されてきた。現在ではCMBの偏光観測が大きなテーマとなっており、偏光を通してインフレーション理論やニュートリノ質量和、宇宙の再電離といった課題に迫ることができる。

GroundBIRDはスペイン領テネリフェ島の標高2,400 mに位置する大角度スケールの観測に特化した地上CMB望遠鏡である。望遠鏡を1分間で20回転させる高速スキャンによって大気揺らぎの影響を抑制した観測を行う。高速スキャンがもたらす効果を最大限に発揮するために、時間応答性の良い超伝導検出器MKIDを焦点面検出器として採用し、2023年5月から本格的な観測を開始した。大角度スケールでの偏光Eモードを観測することで、宇宙再電離期を特徴付け、さらにニュートリノ質量和と縮退したパラメータである光学的厚み$\tau$を誤差$\sigma_{\tau}\sim 0.01$で測定することを目指している。

観測を続けてデータを蓄積する段階にある現在、安定して長期運用をすること、そして質の良いデータを取得することが要求される。しかし、本研究の開始前の観測システムにおいて2つの未解決課題があった。1つは望遠鏡仰角データ取得システムが硬直的であること、もう1つは天球上での検出器配置がスキャン軸から傾いていることである。本論文ではこれらの課題に対する改善と最適化を行なった。

仰角データの取得にFPGAボードを使用しているが、人手を必要とするメンテナンスを高頻度で要し、システム運用をリモート主体で行えないことが、長期運用に対する障壁となっていた。本論文ではボード内のFPGAチップにPYNQと呼ばれるOSシステムを搭載し、OS上からソフトウェアを動かすことでデータ取得の操作性向上とアクセス性の向上を図ることにした。本研究でこのシステムを開発し、望遠鏡にインストールし、信号処理の確認と安定動作を確認した。

焦点面に搭載した多数のMKIDによってCMBの偏光信号を検出する。ここで、無偏光な大気放射ノイズを抑制するためにスキャン軸上の異なるMKID間で信号の差分をとる。しかし、スキャン軸とMKIDの配置軸に傾きがあると差分をとっても大気放射ノイズを十分に差し引けない。本研究では、月の観測データからこの問題を定量的に洗い出した。具体的には、MKIDの配置軸が望遠鏡のスキャン軸に対して約6°傾いていることを見積もった。そして、この結果をもとに、焦点面を含む望遠鏡の構造体を回転させることで天球上でのMKIDの配置軸を改善した。本研究ではさらに月と木星の観測データからこの改善の確認も行なった。加えて、スキャン軸上に配置されているMKID間で信号差分をとり、各入射信号の相関の強さを示す指標に焼き直し、回転の前後で比較することで観測する大気の揺らぎを抑制する結果を得た。

以上2点の改善と最適化を通してGroundBIRDが持つ運用、観測性能の向上を成功させた。



%本論文の構成を述べる。第\ref{chapter1}章でCMBに関わる理論的な背景、第\ref{chapter2}章でGroundBIRD実験の概要を説明する。以降は第\ref{chapter3}章と第\ref{chapter4}章の2部構成になっており、第\ref{chapter3}章でGroundBIRDの角度データ取得システムの改善について、第\ref{chapter4}章で焦点面検出器のアライメント較正とその結果について述べる。第\ref{chapter5}章で今後の展望を述べ、第\ref{chapter6}章でまとめを述べる。


\end{abstract}
