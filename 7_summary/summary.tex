\chapter{まとめ}
\label{chapter6}

CMBの偏光観測は宇宙の進化を説明するための鍵となっており、多くの観測実験が進められている。特に$\ell\sim 10$での大角度スケールのCMB偏光パターンは宇宙の再電離期の情報が刻まれ、その観測によってニュートリノ質量和の精密測定に寄与する重要なプローブである。GroundBIRD実験は大角度スケールのCMB偏光の観測に特化した地上CMB望遠鏡である。望遠鏡を最大で1分間で20回転させる独自のスキャン戦略をとることで、地上実験にとって障壁となる大気放射の揺らぎを抑制した観測を実現する。時間応答性の良い超伝導検出器MKIDを採用し、高速スキャンに伴う高いサンプリングレートを可能にしている。2023年5月に全焦点面検出器のインストールが完了し、GroundBIRDは本格的な物理観測の段階に突入した。
