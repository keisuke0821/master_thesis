\chapter{序論}
\label{chapter1}
宇宙マイクロ波背景放射(Cosmic Microwave Background; CMB)は我々が観測できる宇宙最古の光であり、宇宙初期を知る重要な手がかりとなっている。CMBが持つ温度異方性の観測を基に現代宇宙論の基礎が作られた。しかし、宇宙初期には現在の理論では説明できない物理現象が存在し、これを説明する有力な理論としてインフレーション理論が提唱されている。CMBの偏光にインフレーションの痕跡が残ると考えられており、様々なCMB観測実験が始動している。また、CMBの偏光観測はニュートリノ質量和に対する制限を与えられ、素粒子物理学にも大きな影響を持つ。この章では、CMBとそれを取り巻く宇宙論の関係について述べる。

\section{CMBの異方性と現代宇宙論}

\subsection{CMBの温度異方性}

\subsection{$\Lambda$-CDM~モデル}

\subsection{地平線問題}

\section{CMBの偏光とインフレーション理論}

\subsection{インフレーション理論}

\subsection{CMBの偏光モード}

\subsection{偏光Bモードの探索状況}

\section{CMBの偏光とニュートリノ質量和}

\subsection{光学的厚み~$\tau$}

\subsection{ニュートリノ質量和との縮退}

\subsection{偏光Eモードと$\tau$}
\label{E_and_tau}
