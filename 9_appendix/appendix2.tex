\chapter{読み出し~RF~を用いた~MKID~の応答性の評価}

本文では、MKID~の温度を変えることで応答性を求めた。
ここでは、沓間氏によって開発された読み出し~RF~を用いた応答性の評価手法\cite{kutsuma1}の概要と、
それを発展させた超伝導転移温度の新たな測定手法について述べる。
超伝導転移温度の新たな測定手法は沓間氏と共同で開発した\cite{kutsuma2}。

読み出し~RF~を~MKID~に流すと、そのパワーに応じた準粒子数が~MKID~の内部に生成される。
そのパワーを変えた時の位相変化の時定数を測定することで応答性を求めることができる。
これは、位相変化の時定数が準粒子寿命($\tau_{qp}$)に対応しており、
式~(\ref{tauqp})~を用いることで、準粒子数の変化を求めることができるからある(図~\ref{res_pread})。
\begin{figure}[htbp]
  \centering
  \includegraphics[width=0.8\columnwidth]{9_appendix/figs/res_pread.pdf}
  \caption{読み出し信号を用いた応答性の測定原理\cite{kutsuma1}。 
  読み出し信号が急激に変化すると~(上図)~それに応じて準粒子数~(中図)~と位相~(下図)~が変化する。
  位相の変化時間の時定数が準粒子寿命と対応しており、そこから準粒子数を求める。}
  \label{res_pread}
\end{figure}
この測定を複数のパワーで行うことで位相の応答性を求めることができる。


また、本手法で求まる様々な読み出しパワーでの準粒子寿命を用いることで、
MKID~固有の準粒子寿命を求めることができる。
本手法は~readout~ノイズや~TLS~ノイズの寄与が大きく~GR~ノイズを正しく評価できない場合でも、
準粒子寿命を求めることできる。
このようにして得た~MKID~固有の準粒子寿命から式~(\ref{tauqp})~を用いて超伝導転移温度を求めることができ、
MKID~の温度を実際に変えた際の透過率の変化から求めた超伝導転移温度(図~\ref{res_pread})と一致することを確認し、本手法の妥当性を確認した。
この手法を用いると、MKID~の温度を変えずに準粒子寿命・超伝導転移温度を測定でき、素早いキャリブレーションを行うことができる。

\begin{figure}[htbp]
  \centering
  \includegraphics[width=0.8\columnwidth]{9_appendix/figs/Tc4.eps}
  \caption{MKID~の温度を変化させた時の透過率($S_{21}$)の測定結果。
  1.27~K~付近で透過率が下がることは、アルミニウムが常伝導状態に変化したことを表しており、
  超伝導転移温度であることを示している。}
  \label{res_pread}
\end{figure}

%\newpage
%\chapter{MKID~forecaster}

%観測サイトにおけるシミュレーションに用いた~MKID~forecaster~の概要について述べる。