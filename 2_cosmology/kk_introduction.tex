\chapter{序論}
\label{chapter1}
宇宙マイクロ波背景放射(Cosmic Microwave Background; CMB)は我々が観測できる宇宙最古の光であり、宇宙初期を知る重要な手がかりとなっている。CMBが持つ温度異方性の観測を基に現代宇宙論の基礎が作られた。しかし、宇宙初期には現在の理論では説明できない物理現象が存在し、これを説明する有力な理論としてインフレーション理論が提唱されている。CMBの偏光にインフレーションの痕跡が残ると考えられており、様々なCMB観測実験が始動している。また、CMBの偏光観測はニュートリノ質量和に対する制限を与えられ、素粒子物理学にも大きな影響を持つ。この章では、CMBとそれを取り巻く宇宙論の関係について述べる。

\section{CMBの異方性と現代宇宙論}
ビッグバン理論は宇宙初期が高温高密度であり、膨張しながら星や銀河を作り、今に至るという宇宙のシナリオを予言した。ビッグバンの証拠には宇宙膨張を示すハッブルの法則やビッグバン元素合成(Big Bang Nucleosynthesis; BBN\cite{BBN})と呼ばれる初期宇宙の軽元素の生成過程が挙げられる。そしてもう1つの証拠はCMBの周波数スペクトルがほぼ\SI{2.725}{K}の黒体放射のスペクトルと一致するという観測事実\cite{2725}である。このことで宇宙初期は熱平衡状態だったことが証明された。

宇宙初期は高温高密度であり、バリオン物質がイオン化しており、

\subsection{CMBの温度異方性}

\subsection{$\Lambda$-CDM~モデル}

\subsection{地平線問題}

\section{CMBの偏光とインフレーション理論}

\subsection{インフレーション理論}

\subsection{CMBの偏光モード}

\subsection{偏光Bモードの探索状況}

\section{CMBの偏光とニュートリノ質量和}

\subsection{光学的厚み~$\tau$}

\subsection{ニュートリノ質量和との縮退}

\subsection{偏光Eモードと$\tau$}
\label{E_and_tau}
