\chapter{GroundBIRD実験}

CMB観測実験には地上から観測する実験と衛星を用いて宇宙から観測する実験に分けられる。ここでは私が参加しているGroundBIRD実験(図\ref{GB_overview})について実験の概要と現在の観測状況について説明する。

\begin{figure}[htbp]
  \centering
  \includegraphics[width=0.5\columnwidth]{3_GB/figs/GB_overview2.pdf}
  \caption{GroundBIRD望遠鏡の外観。望遠鏡クライオスタットが方位角回転台の上に設置されており、回転台とともに最大で20RPM(1分間で20回転)の速度で回転する。}
  \label{GB_overview}
\end{figure}
\section{実験概要}

\subsection{GroundBIRD望遠鏡とスキャン戦略}

\subsection{物理ターゲット}

\section{現在の観測状況}

\subsection{検出器のフルアレイインストール}

\subsection{リモート観測システム}
