\begin{thebibliography}{99}
\addcontentsline{toc}{chapter}{\protect\numberline{}{参考文献}\hfil}
% \addcontentsline{toc}{chapter}{\protect\numberline{}{参考文献}}
\markboth{\bibname}{参考文献}

% 2_introduction
\bibitem{BBN}
Wagoner, Robert V. ; Fowler, William A. ; Hoyle, F., Astrophysical Journal, vol. 148, p.3 (1967)
\href{https://doi.org/10.1086/149126}{https://doi.org/10.1086/149126}

\bibitem{2725}
J. C. Mather et al 1999 ApJ 512 511,
\href{https://doi.org/10.1086/306805}{https://doi.org/10.1086/306805}

\bibitem{Planck_T}
Planck Collaboration, A$\&$A 641, A1 (2020),
\href{https://doi.org/10.1051/0004-6361/201833880}{https://doi.org/10.1051/0004-6361/201833880}

\bibitem{nyuumonn}
バーバラ・ライデン[著]. 牧野伸義[訳]. 宇宙論入門. 森北出版

% 3_GB
\bibitem{PWV}
Julio A. Castro-Almaz\'{a}n, Casiana Mu\~{n}oz-Tu\~{n}\'{o}n, Bego\~{n}a Garc\'{i}a-Lorenzo, Gabriel P\'{e}rez-Jord\'{a}n, Antonia M. Varela, and Ignacio Romero ”Precipitable Water Vapour at the Canarian Observatories (Teide and Roque de los Muchachos) from routine GPS”, Proc. SPIE 9910, Observatory Operations: Strategies, Processes, and Systems VI, 99100P (18 July 2016)
\href{https://doi.org/10.1117/12.2232646}{https://doi.org/10.1117/12.2232646}

\bibitem{TES_res}
P. A. R. Ade et al 2015 ApJ 812 176
\href{https://doi.org/10.1088/0004-637X/812/2/176}{https://doi.org/10.1088/0004-637X/812/2/176}

\bibitem{MKID_res}
P. K. Day, H. G. LeDuc, B. A. Mazin, A. Vayonakis, J. Zmuidzinas, Nature 425,
817-821 (2003)
\href{https://doi.org/10.1038/nature02037}{https://doi.org/10.1038/nature02037}

\bibitem{MKID_pic}
\href{https://pubs.aip.org/aip/apl/article-abstract/103/20/203503/130528/High-optical-efficiency-and-photon-noise-limited?redirectedFrom=fulltext}{R. M. J. Janssen, et al., High optical efficiency and photon noise limited sensitivity of microwave kinetic inductance detectors using phase readout. Appl. Phys. Lett. 103, 203503, 2013.}

\bibitem{bcs}
\href{https://journals.aps.org/pr/abstract/10.1103/PhysRev.108.1175}
{J. Bardeen, L. N. Cooper, and J. R. Schrieffer. Theory of superconductivity. Physical Review, 108:1175, 1957.}

\bibitem{tau_measure}
\href{https://lambda.gsfc.nasa.gov/education/graphic\_history/taureionzation.html}{https://lambda.gsfc.nasa.gov/education/graphic\_history/taureionzation.html}

\bibitem{CLASS}
arXiv:2309.00675 [astro-ph.CO]
\href{https://doi.org/10.48550/arXiv.2309.00675}{https://doi.org/10.48550/arXiv.2309.00675}

\bibitem{QUIJOTE}
Monthly Notices of the Royal Astronomical Society, Volume 519, Issue 3, March 2023, Pages 3383—3431
\href{https://doi.org/10.1093/mnras/stac3439}{https://doi.org/10.1093/mnras/stac3439}

\bibitem{spie_honda}
\href{https://www.spiedigitallibrary.org/conference-proceedings-of-spie/11445/114457Q/On-site-performance-of-GroundBIRD-a-CMB-polarization-telescope-for/10.1117/12.2560918.short}
{S.Honda, et al., On-site performance of GroundBIRD, a CMB polarization telescope for large angular scale observations. Proceedings Volume 11445, Ground-based and Airborne Telescopes VIII; 114457Q (2020)}

\bibitem{planck_cmb}
Planck Collaboration, A$\&$A 641, A4 (2020)
\href{https://doi.org/10.1051/0004-6361/201833881}{https://doi.org/10.1051/0004-6361/201833881}

\bibitem{joint_ana}
K. Lee et al 2021 ApJ 915 88
\href{https://doi.org/10.3847/1538-4357/ac024b}{https://doi.org/10.3847/1538-4357/ac024b}

\bibitem{sueno_paper}
\href{https://ui.adsabs.harvard.edu/abs/2024PTEP.2024b3F01S/abstract}
{Y.Sueno, et al., Pointing Calibration of GroundBIRD Telescope Using Moon Observation Data}

\bibitem{sueno_doctor}
末野慶徳. Development of calibration and noise characterization methods for a CMB telecope, GroundBIRD, using its commissioning observation data. 京都大学理学研究科 博士論文 2024.

\bibitem{choi_doctor}
J. Choi, GroundBIRD: A Telescope for the Cosmic Microwave Background Polarization Measurement, ph.D thesis, Korea University (2015).

% 4_elDAQ
\bibitem{R-1SL}
\href{https://canon.jp/biz/product/indtech/incremental-encoder/lineup/r1sl}{https://canon.jp/biz/product/indtech/incremental-encoder/lineup/r1sl}

\bibitem{Zybo}
\href{https://digilent.com/reference/programmable-logic/zybo/start?redirect=1}{https://digilent.com/reference/programmable-logic/zybo/start?redirect=1}

\bibitem{ERM220}
\href{https://www.heidenhain.co.jp/製品/角度エンコーダ/組込み型角度エンコーダ/erm-2000シリーズ}{https://www.heidenhain.co.jp/製品/角度エンコーダ/組込み型角度エンコーダ/erm-2000シリーズ}

\bibitem{Spartan}
\href{https://japan.xilinx.com/support/documentation-navigation/silicon-devices/mature-products/spartan-3e.html}{
https://japan.xilinx.com/support/documentation-navigation/silicon-devices/mature-products/spartan-3e.html}

\bibitem{ikemitsu}
池満拓司. CMB望遠鏡のデータ読み出しシステムの時刻同期と較正に関する開発研究. 京都大学理学研究科 修士論文 2020.

\bibitem{Zynq}
\href{https://docs.amd.com/v/u/en-US/ds187-XC7Z010-XC7Z020-Data-Sheet}{https://docs.amd.com/v/u/en-US/ds187-XC7Z010-XC7Z020-Data-Sheet}

\bibitem{FIFO}
\href{https://japan.xilinx.com/products/intellectual-property/axi\_fifo.html}{https://japan.xilinx.com/products/intellectual-property/axi\_fifo.html}

\bibitem{Pynq}
\href{http://www.pynq.io}{http://www.pynq.io}

\bibitem{image}
\href{https://wasa-labo.com/wp/?p=1102}{https://wasa-labo.com/wp/?p=1102}

\bibitem{power_ref}
\href{https://digilent.com/reference/programmable-logic/zybo-z7/reference-manual?redirect=1}{https://digilent.com/reference/programmable-logic/zybo-z7/reference-manual?redirect=1}

%\bibitem{xadc}
%\href{https://japan.xilinx.com/products/intellectual-property/axi\_xadc.html}{https://japan.xilinx.com/products/intellectual-property/axi\_xadc.html}

\bibitem{xadc}
\href{http://www.kumikomi.net/fpga/sample/0008/FPGA08\_042.pdf}{http://www.kumikomi.net/fpga/sample/0008/FPGA08\_042.pdf}

% 5_alignment

\bibitem{atmos_radiation}
M. A. Janssen, Atmospheric Remote Sensing by Microwave Radiometry (1993)
%\href{https://www.researchgate.net/publication/216681394\_Atmospheric\_Remote\_Sensing\_by\_Microwave\_Radiometry}{https://www.researchgate.net/publication/216681394\_Atmospheric\_Remote\_Sensing\_by\_Microwave\_Radiometry}

\bibitem{dennpa_tennmonn}
中井直正、坪井昌人、福井康雄. シリーズ  現在の天文学. 宇宙の観測I\hspace{-1.2pt}I  電波天文学. 日本評論社

\bibitem{sueno_master}
末野慶徳. 超伝導検出器MKIDの評価系構築とTLSノイズを抑制する研究. 京都大学理学研究科 修士論文 2021.

\bibitem{muto}
武藤優真. 超伝導検出器MKIDの薄膜純度向上及び高感度化を目指した製作と性能評価. 京都大学理学研究科 修士論文 2024.

\bibitem{astropy}
\href{https://www.astropy.org}{https://www.astropy.org}

\bibitem{healpy}
\href{https://healpy.readthedocs.io/en/latest/}{https://healpy.readthedocs.io/en/latest/}

\bibitem{periodogram}
\href{https://docs.scipy.org/doc/scipy/reference/generated/scipy.signal.periodogram.html}{https://docs.scipy.org/doc/scipy/reference/generated/scipy.signal.periodogram.html}


\end{thebibliography}
