\chapter{焦点面検出器アライメントの較正}
\label{chapter4}

GroundBIRD実験での偏光測定のためには、検出器間での信号の差分を取ることが重要であり、それに伴って望遠鏡のスキャンに対して最適な検出器のアライメントが求められる。この章では従来の検出器アライメントの最適化に向けた改善を行い、観測データからその効果を確認した。

\section{検出器アライメントの問題点}

\subsection{スキャン軸に対する傾きと差分解析}

\subsection{要求されるアライメント性能}

\subsection{視線軸方向まわりの回転による較正}

\section{月を用いた回転角の算出}

\subsection{月を用いた理由}

\subsection{必要な回転角}

\subsection{回転する上でのジグの必要性}

\section{ジグの設計と現地インストール}

\subsection{固定用ジグの作成}

\subsection{望遠鏡への実装}

\section{天体を用いた較正結果の確認}

\subsection{月データによる確認}

\subsection{木星データによる確認}

\section{検出器間差分で見る大気揺らぎの抑制}

\subsection{timing offsetの算出}

\subsection{差分解析による確認}
