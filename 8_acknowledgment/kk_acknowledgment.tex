\chapter{謝辞}

京都大学高エネルギー物理学研究室で過ごした2年間の研究生活は大変有意義なもので、多くのことを学び、成長することができたと実感しています。本論文の執筆に至るまでご指導・ご支援いただいた全ての方々に感謝を申し上げます。

田島治教授にはCMBの研究を始める入り口を作っていただき、充実した研究をするきっかけを与えてくださいました。また、ミーティングなどで的確なアドバイスをいただき、研究に対する理解を深めることができました。そして、修士論文を執筆するまで計2回のテネリフェへの出張をすることができたのは田島さんの多大なる協力のお陰です。ご多忙の中、修士論文の添削に多くのお時間を割いていただきありがとうございました。

鈴木惇也助教とは1年間同室で過ごし、多くのご指導とアドバイスをいただきました。日頃から気にかけていただき、基礎的なことから実験の詳細のことまで丁寧に教えていただきました。修士一年目の研究が思うように進まなかった時期には、夜遅くまで付きっきりでご指導していただき、心から感謝しています。
%同期の笠井優太郎くんは優しくていい人でした。
